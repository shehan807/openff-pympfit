\documentclass{article}
\usepackage{log_2022}           

\usepackage{booktabs}            % professional-quality tables
\usepackage{multirow}            % tabular cells spanning multiple rows
\usepackage{amsfonts}            % blackboard math symbols
\usepackage{graphicx}            % figures
\usepackage{duckuments}          % sample images
\usepackage{subcaption}
\usepackage{float}      % for [H] placement

% If you want to use natbib:
\usepackage[numbers,compress,sort]{natbib}
%                                % for numerical citations
% \usepackage[sort,round]{natbib}
%                                % for textual citations

% If you want to use bibLaTeX, uncomment statements below:
% \usepackage[
%      backend=biber,
%      style=numeric-comp,
%      backref=true,
%      natbib=true]{biblatex}
% \addbibresource{reference.bib}

\title[MMomentA: Multipole Moment-based Charge Assignment for Fast, Transferable Polarizable Force Fields]{MMomentA: Multipole Moment-based Charge Assignment for Fast, Transferable Polarizable Force Fields}

\author[S. Parmar et al.]{%
Shehan M. Parmar\\
\institute{Georgia Institute of Technology}\\
\email{sparmar32@gatech.edu}\And
Jesse McDaniel\footnotemark[1]\\
\institute{Georgia Institute of Technology}\\
\email{jesse.mcdaniel@chemistry.gatech.edu}
}


\begin{document}

\maketitle

\begin{abstract}
High-throughput (HT) molecular dynamics (MD) pipelines centrally depend on fast, efficient, and transferable partial atomic charge assignment schemes. Electrostatic potential (ESP) charge methods are widely used and accurate, but costly to generate at scale. Multipole-moment-based schemes (e.g., Gaussian Distributed Multipole Analysis (GDMA), Minimal Basis Iterative Stockholder) significantly improve charge interpretability for polarizable force fields at the expense of computational cost. To this end, we introduce \textit{MMomentA} (Multipole Moment-based Charge Assignment), a graph neural network that learns a charge model informed by GDMA multipoles while enforcing total charge conservation. We benchmark AM1-BCC, MPFIT, and \textit{MMomentA} on a 1,000-molecule subset of the ZINC dataset, evaluating each method’s ability to reproduce ab initio ESPs generated in Psi4; \textit{MMomentA} achieves competitive charge errors with $\sim$14,ms per-molecule inference suitable for HTMD.
\end{abstract}

\section{Introduction}

High-throughput (HT) molecular dynamics (MD) has emerged as a powerful approach for accelerating materials discovery through large-scale simulations of molecular liquids, interfaces, and condensed phases. Modern workflow managers such as JobFlow~\cite{rosen_jobflow_2024}, FireWorks~\cite{jain_fireworks_2015}, and our most recent contribution, atomate2~\cite{ganose_atomate2_2025}, now support engine-agnostic schema for HTMD. This software infrastructure streamlines force field assignment, atom typing, system construction (e.g., Packmol~\cite{martinez_packmol_2009}, pymatgen~\cite{ong_python_2013}), execution of various MD workflows (e.g., energy minimization, \textit{NpT}, \textit{NVT}, annealing, etc.), and trajectory analysis (e.g., MDAnalysis~\cite{michaud-agrawal_mdanalysis_2011}). However, a persistent challenge in deploying HTMD pipelines lies in parameterizing classical force fields, particularly in assigning accurate and transferable partial atomic charges. Because in many condensed-phase and biological materials electrostatic interactions govern macroscopic behavior, charge assignment is critical for accurate estimation of downstream thermophysical property prediction tasks.~\cite{sambasivarao_development_2009, Doerr2016HTMD:Discovery, mcdaniel_development_2014, goloviznina_clpol_2022}

Charge fitting methods have long been a foundational component of classical force fields. Early electrostatic potential (ESP)-based approaches, like CHELPG~\cite{breneman_determining_1990}, fit quantum mechanical (QM)-derived ESPs onto atomic centers via least-squares regression.~\cite{hu_fitting_2007} Methods like the restrained electrostatic potential (RESP) method furthers this approach by enforcing hyperbolic restraints to reduce the overall atomic charge magnitude and maximize transferability.~\cite{wang_psiresp_2022, zhao_pyresp_2022} ESP-based methods present a challenge most severe when ``buried atoms'' are present in a molecule: an undetermined number of charge distributions can reproduce the same ESP (unless in the limit of infinite grid points around a molecule).~\cite{bleiziffer_machine_2018} Physically interpretable partial atomic charges can be derived from atomic multipole moments.~\cite{stone_distributed_1981, ferenczy_charges_1991} For instance, the Gaussian Distributed Multipole Analysis (GDMA) developed by Stone~\cite{stone_distributed_1981} considers multipole moments as a truncated series expansion for which partial charges can better describe anisotropic charge distributions.~\cite{jing_polarizable_2019, chipot_transferable_1993}

Despite these various advancements in charge fitting methods, a persistent tradeoff between computational efficiency and accuracy remains. Moreover, on one hand, ESP-based approaches have shown considerable conformational dependence, whereas GDMA-based approaches are prone to basis-set dependence. Moreover, small deviations in assigned point charges have shown to significantly influence downstream predictions, including surface tension, self-diffusion coefficients, and conductivity.~\cite{ruza_benchmarking_2025, amin_benchmarking_2020, roos_opls3e_2019, kadaoluwa_pathirannahalage_systematic_2021} Thus, modern charge assignment protocols have leveraged large molecular datasets (e.g., ZINC~\cite{irwin_zinc_2005}, SPICE~\cite{eastman_spice_2023}, QM9~\cite{ramakrishnan_quantum_2014}) to generate machine learning models that retain physical interpretability, accuracy, and low-cost.~\cite{wang_espalomacharge_2024, bleiziffer_machine_2018}

To this end, we introduce \textit{MMomentA}, a machine learning (ML) framework that accelerates GDMA-based charge assignment using a graph neural network (GNN). Moreover, we show that, at worst, ML-based multipole analyses reproduce the overall ESP of common charge methods like AM1-BCC~\cite{jakalian_fast_2002}. As a demonstration, in this paper, we train our GNN on a subset of the ZINC~\cite{irwin_zinc_2005} dataset to show a proof-of-concept ML-based charge assignment. With these promising results, we plan to scale up to larger, 100k+ molecular datasets to improve transferability and implementation for HTMD pipelines. 

\section{Background}

\textbf{Gaussian Distributed Multipole Analysis.} The Gaussian Distributed Multipole Analysis (GDMA) introduced by Stone~\cite{stone_distributed_1981} provides a rigorous framework for decomposing a molecular charge density into a series of multipole moments localized on atomic centers. The electrostatic potential at a given point in space, $\mathbf{r}$, can be expressed as a multipole expansion,

\begin{equation}
    V(\mathbf{r}) = \frac{1}{4\pi\epsilon_0}\sum_{n=0}^{\infty}\frac{1}{r^{(n+1)}}\int (r')^n P_n(\cos{\alpha})\rho(\mathbf{r'})d\tau',
\end{equation}

where $P_n$ are Legendre polynomials and $\rho(\mathbf{r})$ is an arbitrary, localized charge distribution. Molecular charge distributions are hence defined as 

\begin{equation}
    \rho(\mathbf{r}) = \sum_{ij} D_{ij}\chi_i(\mathbf{r})\chi_j(\mathbf{r}),   
\end{equation}

where $D_{ij}$ is an element of the density matrix, and $\chi(\mathbf{r})$ is a normalized basis function, expressed as a linear combination of gaussian primitive functions, i.e., local multipoles~\cite{stone_distributed_1981} 

\begin{equation}
    \chi_i(\mathbf{r}) = N_ix_i^{a_i}y_i^{b_i}z_i^{c_i}\exp{[-\zeta_i(\mathbf{r_i})^2]},
\end{equation}

where $a_i + b_i + c_i$ are equal to the angular momentum quantum number, $l$, and $N_i$ is basis function coefficient. Evaluating the higher moments of the overlap density distribution (i.e., $\chi_i\chi_j$) yields the multipole moment of order $lm$,

\begin{equation}\label{one_center_moment}
    Q_{lm} = \int R_{lm}(\mathbf{r})\rho(\mathbf{r})d^3\mathbf{r},
\end{equation}

where $R_{lm}$ is a regular solid harmonic.~\cite{ferenczy_charges_1991} GDMA expands Equation \ref{one_center_moment} by \textit{distributing} such moments across atoms and bonds (or even arbitrary virtual sites), serving as a physically interpretable bridge between QM charge density and classical force field electrostatics.

\subsection{Machine Learning for Partial Charge Assignment.} Recent advancements in machine learning (ML) have led to breakthroughs in partial charge assignment by reducing computational cost while maintaining quantum-mechanical (QM) accuracy. Early models such as PhysNet~\cite{unke_physnet_2019} and SchNet~\cite{schutt_schnet_2017} demonstrated that message-passing neural networks can predict atom-centered properties, including partial charges, from local chemical environments. More specialized frameworks have since emerged. For instance, EspalomaCharge embeds charge equilibration (QEq) directly into a graph neural network (GNN) to predict atomic electronegativities and hardness parameters while preserving total molecular charge~\cite{wang_espalomacharge_2024}. The Riniker group and others~\cite{bleiziffer_machine_2018} have reproduced AM1-BCC charges within $\pm 0.01\:e$ accuracy, achieving order-of-magnitude speedups compared to density-derived electrostatic and chemical (DDEC) methods. To the best of the authors’ knowledge, GDMA-based charge assignment methods have not yet been integrated with modern GNN architectures, leaving significant room for improvement in computational speed, interpretability, and transferability.

\section{Methods}

\subsection{Multipole Fitting}

The MPFIT (multipole fitting) procedure originally outlined by Ferenczy~\cite{ferenczy_charges_1991} represents the traditional approach to deriving atom-centered partial charges from GDMA output. Formally, this is achieved by minimizing the difference between the potential created by the distributed multipole moments, $V^{\text{GDMA}}(\mathbf{r})$, and that generated by the fitted charges, $V^{Q}(\mathbf{r})$,

\begin{equation}
f(\mathbf{r}) = V^{\text{GDMA}}(\mathbf{r}) - V^{Q}(\mathbf{r}) = \sum_{a}\sum_{l,m} Q_{lm}^{a}I_{lm}^{a}(\mathbf{r}) - \sum_{i} q_i I_{00}^{i}(\mathbf{r}),
\end{equation}

where $Q_{lm}^{a}$ denotes the $m$th component of the rank-$l$ multipole moment centered at site $a$, $q_i$ are the point charges, and $I_{lm}^{a}(\mathbf{r}) = r_a^{-(l+1)}C_{lm}(\theta,\phi)$ are irregular solid harmonics. The optimal charges are obtained by minimizing the integrated squared error,

\begin{equation}
\frac{\delta}{\delta q_j^a} \int [f(\mathbf{r})]^2 r^2 \sin\theta,dr,d\theta,d\phi = 0,
\end{equation}

which yields a linear system of the form $Aq^a = b$, where the elements of $A$ and $b$ depend on regular and irregular solid harmonics across atomic centers and multipole ranks. Solving for $q^a = A^{-1}b$ yields the set of least-squares charges that best reproduce the distributed multipole potential. In this work, the openff-PyMPFIT (\href{https://github.com/shehan807/openff-PyMPFIT}{https://github.com/shehan807/openff-PyMPFIT}) implementation integrated with the Open Force Field Toolkit~\cite{wang_open_2024} was used. For each molecule, QM calculations were performed in Psi4~\cite{smith_psi4_2020} at the HF/6-31G* level of theory, followed by a GDMA up to rank 8 ($l_{\rm max} = 8$). 

\subsection{\textit{MMomentA} Architecture}

\textit{MMomentA} (Multipole Moment-based Charge Assignment) employs a graph neural network (GNN) that extends the EspalomaCharge~\cite{wang_espalomacharge_2024} architecture with optional GDMA moment features. The molecular graph, $G = (V, E)$ is comprised of atoms as nodes ($V$) and bonds as edges ($E$). The architecture consists of a GraphSAGE-based message-passing network (4 layers, 128 hidden units per layer) with mean aggregation~\cite{Hamilton:2017tp}, a readout layer that predicts atomic electronegativity ($e_i$) and hardness ($s_i$), and a charge equilibration layer to conserve charge, i.e., $\sum q_i =Q$.

Node features consist of atomic numbers (100-dimensional one-hot encoding) and several other atomic fingerprints used in the EspalomaCharge framework (e.g., formal charge, atomic mass, explicit valence, aromaticity, etc.), totaling an input dimension of 117. Experiments were conducted to show node feature sets augmented by multipole moments of up to rank $l = 8$ made marginal changes in underlying performance. The entire node and edge scheme is functionally permutationally invariant. Given this architecture, \textit{MMomentA} minimizes the mean squared error between the reference, MPFIT charges and the predicted, \textit{MMomentA} charges. The model was implemented in PyTorch 2.0 and DGL 1.1 using the Adam optimizer and learning rate of $\eta = 10^{-3}$. The code for \textit{MMomentA} is provided here: \href{https://github.com/shehan807/MMomentA}{https://github.com/shehan807/MMomentA}

\subsection{Dataset}

The \textit{MMomentA} repository supports dataloaders and preprocessing for three primary datasets: Zinc~\cite{irwin_zinc_2005}, SPICE 2.0.1 (Small-molecule/Protein Interaction Chemical Energies)~\cite{eastman_spice_2023}, and QM9~\cite{ramakrishnan_quantum_2014}. For each dataset, the MMomentA pipeline converts SMILES to molecule objects in OpenFF, performs QM single-point energy calculations in Psi4, and computes the GDMA moments up to $l=8$. Splits are created with an 80/10/10 train/validation/test protocol.

\section{Results and Discussion}

We evaluated three charge methods, AM1-BCC, MPFIT, and \textit{MMomentA}, on a 1000 molecule subset of the ZINC dataset. Training was conducted on an NVIDIA A100 GPU on the NERSC supercomputer. Although RMSEs were used for the loss function between MPFIT and the predicted \textit{MMomentA} GNN, to truly test QM-level accuracy, we compare each method's ability to reproduce the ESP generated by QM in Psi4, as shown in Figure~\ref{fig:violin-mae}. As made evident by the MAE and RMSEs, MPFIT-based point charges reproduce QM ESPs similar to AM1-BCC. Moreover, the \textit{MMomentA} GNN reproduces the QM ESPs with $\sim\:0.046$ a.u. MAEs.  

% Figure: Single violin plot (ESP MAE)
% \begin{figure}[h!]
%   \centering
%   \includegraphics[width=0.5\linewidth]{figures/esp_validation_mae.png}
%   \caption{Total mean absolute errors (MAE) between point charge and QM electrostatic potentials (ESP) on the ZINC test split for AM1-BCC, MPFIT (GDMA reference), and \textit{MMomentA}.}
%   \label{fig:violin-mae}
% \end{figure}

% Figure 1: Side-by-side violin plots (ESP MAE and ESP RMSE)
\begin{figure}[h!]
  \centering
  \begin{subfigure}{0.49\linewidth}
    \centering
    \includegraphics[width=\linewidth]{figures/esp_validation_mae.png}
    \caption{QM vs. Point Charge MAE}
    \label{fig:violin-mae}
  \end{subfigure}
  \hfill
  \begin{subfigure}{0.49\linewidth}
    \centering
    \includegraphics[width=\linewidth]{figures/esp_validation_rmse.png}
    \caption{QM vs. Point Charge ESP RMSE}
    \label{fig:violin-rmse}
  \end{subfigure}
  \caption{Total mean absolute and root mean squared error distributions between point charge and QM electrostatic potentials (ESP) on the ZINC test split for AM1-BCC, MPFIT (GDMA reference), and \textit{MMomentA}.}
  \label{fig:violin}
\end{figure}

Table~\ref{tab:summary} shows the order-of-magnitude computational savings by using \textit{MMomentA} at inference time as opposed to traditional MPFIT methods that are precluded by costly GDMA calculations. 

% Table: Charge errors vs QM ESP (exact dictionary values)
\begin{table}[h!]
  \centering
  \caption{Per-atom \textbf{ESP} errors in \textbf{a.u.} computed against the \textbf{QM ESP} target on the ZINC test split, plus per-molecule timing (\textbf{s}).}
  \label{tab:summary}
  \begin{tabular}{lccc}
    \toprule
    \multirow{2.5}{*}{Method} & \multicolumn{2}{c}{QM ESP Error (a.u.)} & \multicolumn{1}{c}{Efficiency} \\
    \cmidrule(lr){2-3}
    & \(\mathbf{MAE}\) & \(\mathbf{RMSE}\) & \(\mathbf{Time/mol\ (s)}\) \\
    \midrule
    AM1\text{-}BCC & 0.039 & 0.052 & 66.6 \\
    % RESP & \dots & \dots & \dots \\
    MPFIT (GDMA) & \textbf{0.026} & \textbf{0.037} & 91.1 \\
    MMomentA (GNN) & 0.051 & 0.066 & \textbf{0.008} \\
    \bottomrule
  \end{tabular}
\end{table}

The preliminary results are encouraging: \textit{MMomentA} achieves sub–second inference and charge errors that compare with classical baselines on drug-like molecules. Due to time and compute constraints, we report results on a curated \emph{subset} of ZINC to demonstrate proof of concept. Full-scale training on the broader ZINC distribution ($>10^5$ molecules) and the SPICE/QM9 datasets is \textbf{already implemented} in the code infrastructure and is \textbf{in progress}; the repository includes scripts to reproduce data preprocessing, QM, GDMA, MPFIT target generation, and graph featurization. Next, we will (i) scale training and assess generalization across charge states and rarer elements, (ii) benchmark architectural variants (e.g., equivariant layers, GDMA-feature augmentation), and (iii) evaluate downstream sensitivity of transport and interfacial properties (viscosity, diffusion, conductivity, surface tension) to charge models within HTMD pipelines. 
% Finally, we plan to extend the framework to virtual sites and polarizable models, a Bayesian inference-based feature already implemented into the MPFIT repository, enabling fast, transferable electrostatics beyond fixed point charges.

\section{Conclusions}

In this work, we have developed a graph neural network (GNN)-based, multipole moment charge model, \textit{MMomentA}. Compared to classical, ESP-based fitting approaches, \textit{MMomentA} delivers accurate point charges that reproduce QM ESPs at order-of-magnitudes less of the cost of GDMA-based multipole fitting methods. On a 1000 molecule subset of the ZINC dataset, the model attains reasonable MAE/RMSEs against QM ESPs. These results indicate \textit{MMomentA} as a charge model positioned for immediate use in high-throughput molecular dynamics workflows, like atomate2.~\cite{ganose_atomate2_2025} 

% Ongoing work scales training to $>10^5$ ZINC molecules and adds SPICE/QM9 coverage, extends to flexible virtual site assignment, and quantifies sensitivity of downstream predictions to charge methods. 

\section*{Acknowledgements}

This material is based upon work supported by the U.S. Department of Energy, Office of Science, Office of Advanced Scientific Computing Research, Department of Energy Computational Science Graduate Fellowship under Award Number DE-SC0022158 and the NERSC ERCAP 2022 grant number ERCAP0021857. 

% For natbib users:
\bibliographystyle{unsrtnat}
\bibliography{Zotero_MMomentA, reference}
% For bibLaTeX users:
% \printbibliography

% \appendix
% \section{Appendix}
% Any possible appendices should be placed after bibliographies.
% If your paper has appendices, please submit the appendices together with the main body of the paper.
% There will be no separate supplementary material submission.
% The main text should be self-contained; reviewers are not obliged to look at the appendices when writing their review comments.

\end{document}

